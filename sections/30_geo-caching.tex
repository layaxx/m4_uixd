\section{Geo-Caching}

Ein (physischen) Geocache in einer urbanen Umgebung war im Raum Bamberg auf Opencaching.de kaum zu finden, weshalb ich mich für die folgenden Caches entschied, die zwar beide in Parks aber doch recht Nahe am urbanen Raum liegen.

\subsection*{Down in the Jungle!}

Es ist kurz vor 8:00, als ich mich vom alten Rathaus aus auf den Weg mache. Der Geocache, für den ich mich entschieden habe, heißt \enquote{Down in the Jungle!}\footnote{\url{https://www.opencaching.de/viewcache.php?wp=OC130C6}}. Er soll in der Nähe des Klosters St. Michael liegen, zwar im Stadtgebiet, aber doch auf einer Grünfläche.

Ich kopiere die Koordinaten von der Internetseite, füge Sie in der mobilen App von GoogleMaps ein, starte die Navigationsfunktion und betrachte den vorgeschlagenen Weg. Ich laufe los, Richtung Sandstraße. GoogleMaps zeigt eine Routendauer von 20 Minuten an. Ab und zu werfe ich einen Blick auf mein Handy, um sicherzugehen, noch auf der korrekten Route zu sein.

Ich entscheide mich, der Alternativroute in den Grünhundsbrunnen zu folgen. An der nächsten Kreuzung werfe ich erneut einen Blick aufs Handy. Ich habe die Route noch richtig im Kopf, will aber sicher gehen. Etwa die Hälfte der Strecke ist geschafft.

In der Nähe des Klosters scheint mir eine Baustelle den Weg zu versperren. Ein Umweg ist aber nicht notwendig, für Fußgänger ist eine Querung möglich. Damit bin ich raus aus der Stadt, befinde mich in einer Parkanlage. Ein Spaziergänger mit Hund kommt mir entgegen.

Noch ein Blick aufs Handy, neben der Einfahrt auf ein privates Grundstück geht ein Weg ab. Der Name des Caches ist treffen gewählt, man fühlt sich zwischen den Bäumen wirklich fast wie im Jungle. Nur noch Hundert Meter, laut GoogleMaps. Die Markierung auf der Karte zeigt auf einen Knick im Weg, neben einem kleinen Teich steht ein eindrucksvoller Baum. Der Zulauf zum Teich ist mit einem Geländer gesichert.

Ich stecke das Handy weg, dessen Arbeit ist erst einmal getan. Jetzt heißt es: Suchen. Ich begutachte das Geländer, die Rohre beim Zulauf des Teiches, besonders den Baum. Ist hier die Dose versteckt? Möglichkeiten gäbe es auf jeden Fall, der Baum scheint am vielversprechendsten.

Das Herbstlaub erschwert die Suche, mit den Händen schiebe ich es zur Seite, um in Kuhlen und Astgabeln nachzusehen. Die Beschreibung des Caches bietet auch keine weiteren Anhaltspunkte. Die Kommentare geben mir kaum Anlass zur Hoffnung: Der letzte dokumentierte Fund ist bereits drei Jahre her. In der Zeit kann viel passiert sein.

Ich werfe erneut einen Blick auf die Karte: Keine Zweifel, hier muss es sein. Nach circa 25 Minuten breche ich die Suche ab, zu hoch die Wahrscheinlichkeit, dass sich die Dose nicht mehr hier befindet.

\subsection*{Schweinskopf al dente}
Mittags mache ich mich auf den Weg zum zweiten Versuch, diesmal aber an einem anderen Ort. Mit dem Bus geht es Richtung Hain, zum \enquote{Schweinskopf al dente}\footnote{\url{https://www.opencaching.de/viewcache.php?wp=OC141A5}}. Der letzte Eintrag ist hier vom 26. März 2024, die Chancen sollten also ganz gut stehen, dass dieser Cache noch zu finden ist.

Von der Haltestelle Wilhelm-Löhe-Heim aus geht es zu Fuß weiter. Das Prozedere ist dasselbe wie beim ersten Versuch: Koordinaten in GoogleMaps einfügen, Zielführung starten und los geht's.

Erneut entscheide ich mich für eine alternative Route. Der exakte Ort scheint auf der Karte näher an dem direkt am Wasser verlaufenden Weg zu liegen, als an dem Weg, auf den mich die Navigation führen will. Diesem Weg folge ich also noch etwas, dann bin ich angekommen.

Die Koordinaten liegen rechts des Weges, was auch Sinn macht: Links folgt auf die Böschung direkt die Regnitz, hier gibt es kaum Möglichkeiten, eine Box zu verstecken. Rechts hingegen Bäume, Sträucher, Wurzeln.

Also stecke ich wieder mein Handy weg und mache mich auf die Suche. Wieder fällt mein Blick zuerst auf einen kräftigen Baum, ich suche die Wurzeln ab, suche nach Höhlen oder Astgabelungen. Als ich hier nicht fündig werde, streift mein Blick durch die Umgebung. Glücklicherweise sind kaum andere Personen unterwegs, auf die ich Rücksicht nehmen müsste.

Mei Blick bleibt an einem Baumstumpf hängen. Da sollte es doch auch ein paar gute Verstecke geben. Ich umrunden das Objekt, streiche erneut Laub beiseite. Und tatsächlich, bei der zweiten Umrundung werde ich fündig. Am Anfang muss ich den grünen Deckel der Box noch übersehen haben, der nun zwischen Rinde und Laub hervorlugt.

Ich nehme die Box aus ihrem Versteck und öffne sie. Ich blättere durch das etwas feuchte, leicht gammelige Logbuch und trage das Datum und meinen Namen ein. Anschließend lege ich es zurück und platziere die Box an dem Ort, an dem ich sie wenige Minuten vorher gefunden habe.
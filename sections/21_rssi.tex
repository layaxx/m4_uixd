\subsection{RSSI}
\label{sec:rssi}

\subsubsection*{Durchführung}
Die folgenden beiden Aufgaben waren gestalteten sich unerwartet schwierig. Dies lag insbesondere daran, dass für iOS seit längerem keine Apps verfügbar sind, die die Stärke von WiFi Access Points messen und anzeigen können, da Apple den Zugriff auf die entsprechende Hardware einschränkt.

Schließlich entschied ich mich, die Messungen mit meinem Android Tablet durchzuführen. Hierzu installierte ih verschiedene Apps:

\begin{itemize}
    \item \textbf{WiGLE WiFi Wardriving} von WiGLE.net
    \item \textbf{Signal Strength} von Lakshman
    \item \textbf{WiFi Analyzer} von Propane Apps, manchmal auch \enquote{WiFi Data} genannt
\end{itemize}

Die Messungen wurden mit diesen drei Apps auf meinem Samsung Tab S7FE durchgeführt. Dazu habe ich mir einen gut sichtbaren AP im Erdgeschoss des Gebäudes F21 gesucht. Dieser ist an der Wand am Ende eines Ganges befestigt, sodass eine Entfernungsmessung auf viele Meter ohne Hindernisse möglich ist.

Dann habe ich je eine Messreihe pro App durchgeführt. Dafür habe ich mich direkt an dem AP platziert und den Messwert der App notiert. Dann habe ich mich ca. einen Meter vom AP entfernt und den nächsten Messwert notiert. Dieses Vorgehen habe ich für jede der drei Apps wiederholt. Dabei habe ich darauf geachtet, immer denselben AP zu messen, und nicht etwa auf einen anderen, gegebenenfalls näher liegenden, zu wechseln.

\subsubsection*{Auswertung}

Die Messwerte sind in in Grafik \ref{rssi} zu sehen. Im Anhang sind außerdem in Grafik \ref{rssi-facet} die Messreihen der einzelnen Apps dargestellt. Hier kann man gut erkennen, dass die Werte der App Signal Strength dem erwarteten Verlauf am nächsten kommen. Die Messwerte der App \enquote{WiGLE WiFi} scheinen deutlich mehr zu schwanken. Welche App nun die genaueren Werte liefert, lässt sich anhand dieser Messungen aber natürlich nicht beurteilen.

\begin{figure}[h]
    \centering
    \includegraphics[width=0.9\textwidth]{figures/rssi_all.pdf}
    \caption{RSSI-Messwerte der drei Apps auf einen Blick.}
    \label{rssi}
\end{figure}

Bei allen drei Apps ist jedoch ein deutlicher Abfall der Signalstärke mit steigender Distanz zu erkennen. Der in etwa lineare Verlauf passt auch zur Erwartung, wenn man weiß, dass die Signalstärke zwar mit dem Quadrat der Distanz abnimmt, aber normalerweise in einer logarithmischen Einheit (Dezibel oder Dezibel-Milliwatt) gemessen wird. Theoretisch sollten die Messwerte mit steigender Distanz natürlich streng monoton fallen, was hier nicht immer der Fall ist. Dies könnte viele Ursachen haben, von Ungenauigkeiten der zur Messung verwendeten Hard- und Software über schwankende Sendeleistung des AP über die generelle  Umgebung, in der die Messungen durchgeführt wurden.


Auch die Wertebereiche sind in etwa gleich, was darauf hindeutet, dass die Apps die Werte in etwa gleich skalieren.

\subsubsection*{Einschränkungen}
Die Genauigkeit der Messungen ist durch einige Faktoren eingeschränkt. Zum einen ist die Distanzmessung nicht exakt, da ich als Mass nur meine Schritte, in Kombination mit den Fließen des Bodens zur Verfügung hatte. Die Distanz in den Plots kann grob als Meter interpretiert werden. Zu einer genaueren Messung hätte ich ein Maßband oder ähnliches verwenden müssen. außerdem hätte dann auch die senkrechte Entfernung zwischen Tablet und AP einbezogen werden müssen, da dieser auf über 2m Höhe angebracht ist.

Zum anderen wurden die Messreihen nacheinander durchgeführt, sodass sich die Umgebung durch andere Personen oder andere Störquellen verändert haben könnte. Auch die Positionierung des Tablets kann nicht exakt reproduziert werden, sodass die Messungen nicht exakt vergleichbar sind.

Wie die Plots zeigen, ist das Ergebnis aber dennoch in etwa wie erwartet, sodass die Messungen als erfolgreich betrachtet werden können.

